\documentclass[lang=en,10pt]{elegantbook}

% --- Informasi Metadata Buku ---
\title{Laporan Validasi Pensiun}
\subtitle{Log Percobaan Cek Balance}

\author{Muhammad Zaki Zulhamlizar}
\institute{Universitas Indonesia}
\date{\today}
\version{1.0}

% Info bio kustom (Kiri: Judul Label, Kanan: Isi)
\bioinfo{Info Khusus}{Percobaan Validasi Pensiun dengan \\ Cek Balance banyak eksperimen dan hasil yang menarik.}

% Kutipan atau info tambahan di bawah info penulis
\extrainfo{Mata Kuliah Teori Dana Pensiun.}

% Kedalaman daftar isi (3 artinya sampai sub-subsection)
\setcounter{tocdepth}{3}

% --- Gambar Sampul dan Logo ---
% Pastikan file gambar ini ada di folder project Anda
\logo{muscle-cat.png}
\cover{cover.jpg}

% --- Perintah Dokumen ---
\usepackage{array}
\newcommand{\ccr}[1]{\makecell{{\color{#1}\rule{1cm}{1cm}}}}
\graphicspath{{figure/}}

\renewcommand{\contentsname}{Daftar Isi}
\renewcommand{\chaptername}{Bab}
\renewcommand{\figurename}{Gambar}

% --- Kustomisasi Warna ---
% Ubah warna pita oranye pada halaman judul (opsional, hilangkan komentar untuk mengaktifkan)
\definecolor{customcolor}{RGB}{32,178,170}
\colorlet{coverlinecolor}{customcolor}

\begin{document}

% --- Bagian Depan (Frontmatter) ---
\maketitle      % Membuat halaman judul
\frontmatter    % Penomoran halaman romawi

\tableofcontents % Membuat Daftar Isi

% --- Bagian Utama (Mainmatter) ---
\mainmatter     % Penomoran halaman arab (1, 2, 3...)

% =====================================================
% MULAI MENULIS DI SINI
% =====================================================

\chapter{Brainstorming dulu}

Apa sih yang gw lakuin?\\
Jadi, gw ngelakuin eksperimen buat validasi dana pensiun pake metode cek balance. Jadi, gw ngecek balance dari dana pensiun yang ada buat liat apakah sesuai sama ekspektasi atau enggak.

\vspace{0.5cm}
Jadi, ngapain? \\
Spesifiknya gw akan melakukan validasi angka-angka pada Slide 17 (``Permasalahan Program Jaminan Pensiun'') dan Slide 16 (``Permasalahan Program Jaminan Hari Tua''), kita perlu membedah ``mesin'' aktuaria di baliknya.

\vspace{0.25cm}

Gw akan bandingkan dua sisi neraca:
\begin{itemize}
    \item \textbf{Sisi Aset (Accumulated Fund)}: Berapa uang yang terkumpul dari iuran (mirip prinsip Defined Contribution).
    \item \textbf{Sisi Aset (Accumulated Fund)}: Berapa uang yang seharusnya dibayar ke peserta berdasarkan perhitungan aktuaria (mirip prinsip Defined Benefit).
\end{itemize}

Kalau kedua sisi ini nggak balance, berarti ada yang salah di perhitungan atau asumsi yang dipakai.

\vspace{0.25cm}

Ketimpangan antara kedua sisi ini bisa ngasih insight tentang:
\begin{itemize}
    \item Apakah iuran yang dikumpulkan cukup buat bayar manfaat pensiun.
    \item Apakah asumsi aktuaria (seperti tingkat bunga, mortalitas, dsb) realistis.
    \item Potensi risiko pendanaan jangka panjang dari program pensiun tersebut.
    \item Ketimpangan (Unfunded Liability) yang mungkin ada.
\end{itemize}

\vspace{0.1cm}

Dengan ngecek balance ini, kita bisa validasi apakah program pensiun itu sehat secara finansial atau perlu penyesuaian.

\vspace{0.25cm}

% --- Bab 1 Decostruction & Theory ---
Sebelum masuk ke eksperimen, gw harus menetapkan asumsi matematis berdasarkan teori aktuaria standar yang kemungkinan besar digunakan oleh BKF (Badan Kebijakan Fiskal) dalam slide tersebut.

\section{Validasi Jaminan Pensiun}

Gw mulai dengan validasi jaminan pensiun. Berdasarkan slide 17, gw asumsikan beberapa parameter dasar:
\begin{itemize}
    \item Tingkat bunga diskonto: 7\% per tahun
    \item Mortalitas: Tabel mortalitas standar (misalnya TMI-2019)
    \item Iuran: 3\% dari gaji peserta
    \item Manfaat pensiun: Dihitung berdasarkan rumus tertentu (misalnya 1.5\% per tahun masa kerja)
    \item usia pensiun: 58 tahun
    \item usia masuk kerja: 30 tahun
    \item periode pengamatan: 30 tahun
    \item Frekuensi pembayaran: tahunan
    \item Distribusi gaji: Pertumbuhan gaji 5\% per tahun
    \item Formula Manfaat (Juli 2015 - Sekarang):
\end{itemize}
    
    \begin{equation}
        \text{Benefit} = 1\% \times \text{Masa Iur} \times \text{Rata-rata Upah tertimbang}
    \end{equation}

\begin{itemize}
    \item \textbf{Valuasi Liabilitas:} Nilai Manfaat di tabel slide adalah Actuarial Present Value (APV) dari anuitas seumur hidup pada usia pensiun.
\end{itemize}

% Persamaan hanya menarik 12 tahun iuran
\begin{equation}
    APV = \text{Manfaat Tahunan} \times \ddot{a}_{x}^{12}
\end{equation}

\begin{itemize}
    \item \textbf{Valuasi Aset:} Akumulasi iuran (3\% dari upah) yang dikembangkan dengan tingkat pengembalian investasi (investment return).
\end{itemize}

\begin{equation}
    FV = \sum_{t=1}^{n} \text{Iuran Tahunan} \times (1 + r)^{n-t}
\end{equation}

\vspace{0.5cm}

\section{Implementasi Eksperimen}

gw buat skrip python untuk ngitung aktuaria untuk reverse engineering nilai-nilai di slide. gw taro di `src/'.

\vspace{0.5cm}

Angka pada Slide 17 sangat bergantung pada asumsi makro ekonomi jangka panjang yang tidak ditulis eksplisit di slide.
\begin{itemize}
    \item Jika Salary Increase tinggi, Nilai Manfaat akan meledak (karena rata-rata upah naik).
    \item Jika Investment Return rendah, Akumulasi Dana akan mengecil.
    \item Ketimpangan yang masif di Slide 17 (-311 juta unfunded untuk gaji 2.5 juta) mengindikasikan bahwa asumsi \textbf{Kenaikan Gaji $>$ Return Investasi} atau \textbf{Discount Rate Liabilitas sangat rendah} (sangat konservatif).
\end{itemize}

\chapter{Menganalisis Slide 17}

\section{Definisi Masalah dan Konteks}

Fokus utama penelitian ini berpusat pada \textbf{Slide 17} dari dokumen presentasi yang diterbitkan oleh BKF (\textit{Badan Kebijakan Fiskal}), sebuah unit di bawah Kementerian Keuangan Republik Indonesia. Slide ini memuat data yang sangat krusial sekaligus mengkhawatirkan mengenai kondisi Program Jaminan Pensiun (JP).

Data pada slide tersebut menunjukkan sebuah tabel simulasi di mana seorang pekerja dengan gaji awal Rp 2.500.000, setelah bekerja selama 32 tahun, akan menghasilkan akumulasi dana (aset) sekitar \textbf{Rp 249 Juta}. Namun, di sisi lain, nilai tunai dari janji manfaat pensiun yang harus dibayarkan (liabilitas) tercatat sebesar \textbf{Rp 561 Juta}.

Ketimpangan ini menghasilkan defisit atau \textit{unfunded liability} sebesar \textbf{Rp 311 Juta}.

\textbf{Tujuan Penelitian:}
Saya melakukan penelitian ini untuk memvalidasi angka-angka tersebut. Pertanyaan besarnya adalah:
\begin{enumerate}
    \item Apakah angka defisit masif tersebut valid secara matematis, atau hanya angka yang dibesar-besarkan?
    \item Bagaimana BKF bisa mendapatkan angka Liabilitas sebesar Rp 561 Juta, padahal iuran yang terkumpul hanya Rp 249 Juta?
    \item Asumsi-asumsi tersembunyi apa yang digunakan dalam perhitungan tersebut?
\end{enumerate}

\section{Metodologi Penelitian}

Untuk menjawab pertanyaan di atas, saya tidak bisa hanya menebak. Saya menggunakan dua pendekatan disiplin ilmu secara simultan:

\begin{itemize}
    \item \textbf{Pendekatan Algoritma (Computational Solver):} Menggunakan bahasa pemrograman Python untuk melakukan \textit{Reverse Engineering}. Saya meminta komputer untuk "mencari" kombinasi parameter input yang bisa menghasilkan output sama persis dengan angka BKF.
    \item \textbf{Pendekatan Matematika Aktuaria:} Membedah rumus-rumus valuasi dana pensiun, mulai dari \textit{Time Value of Money} hingga probabilitas mortalitas (\textit{Life Contingencies}), untuk memastikan logika di balik kode Python tersebut sesuai dengan standar aktuaria internasional.
\end{itemize}

\section{Eksperimen Tahap 1: Naive Approach}

\subsection{Ekspektasi Awal}
Pada awalnya, saya berasumsi bahwa perhitungan ini menggunakan standar aktuaria sederhana (\textit{Naive Approach}). Ekspektasi saya adalah:
\begin{itemize}
    \item Pensiun dihitung hanya untuk peserta sendiri (\textit{Single Life}).
    \item Asumsi ekonomi standar: Hasil investasi pasti lebih besar dari kenaikan gaji (agar dana pensiun sehat).
    \item Menggunakan tabel mortalitas standar Indonesia (TMI IV).
\end{itemize}

\subsection{Realita Hasil Simulasi Awal}
Saya menjalankan simulasi Python pertama menggunakan asumsi sederhana di atas. Hasilnya ternyata \textbf{jauh berbeda} dari data BKF.

\begin{table}[htbp]
  \centering
  \caption{Disparitas Hasil Simulasi Awal vs Data BKF}
  \begin{tabular}{lrrr}
    \toprule
    \textbf{Komponen} & \textbf{Simulasi Saya} & \textbf{Data BKF (Slide 17)} & \textbf{Selisih} \\
    \midrule
    Total Aset & Rp 151.960.069 & Rp 249.783.000 & -39\% \\
    Total Liabilitas & Rp 318.986.765 & Rp 561.752.000 & -43\% \\
    \bottomrule
  \end{tabular}
\end{table}

\subsection{Analisis Kegagalan}
Perbedaan yang sangat signifikan ini (hampir separuh dari nilai target) menunjukkan bahwa pemahaman awal saya salah.
\begin{enumerate}
    \item \textbf{Sisi Aset (Rp 151 Jt vs Rp 249 Jt):} Aset hitungan saya terlalu kecil. Ini berarti BKF mengasumsikan gaji peserta naik sangat cepat (\textit{High Salary Growth}), sehingga iuran nominal yang masuk jauh lebih besar.
    \item \textbf{Sisi Liabilitas (Rp 318 Jt vs Rp 561 Jt):} Liabilitas hitungan saya juga terlalu kecil. Ini indikasi kuat bahwa BKF tidak menggunakan asumsi \textit{Single Life}. Dalam regulasi Jaminan Pensiun, manfaat tidak berhenti saat peserta meninggal, melainkan diteruskan ke pasangan (Janda/Duda). Mengabaikan faktor pasangan ini membuat perhitungan liabilitas menjadi \textit{under-valued}.
\end{enumerate}

\section{Eksperimen Tahap 2: Rigorous Approach}

Belajar dari kesalahan di tahap pertama, saya merevisi model matematika dan algoritma saya.

\subsection{Perbaikan Model Aktuaria}
Saya mengubah rumus valuasi liabilitas dari \textit{Single Life Annuity} ($\ddot{a}_x$) menjadi \textbf{\textit{Joint Life Reversionary Annuity}}.

Secara intuitif, model ini berkata: "Dana pensiun harus cukup untuk membiayai peserta seumur hidup, \textbf{DAN} jika peserta meninggal, dana harus cukup untuk membiayai pasangannya sebesar 50\% dari manfaat."

\subsection{Perbaikan Parameter Ekonomi (Forensik Data)}
Menggunakan algoritma \textit{solver} Python, saya mencari parameter ekonomi yang bisa mencocokkan angka aset dan liabilitas sekaligus. Ditemukan sebuah fenomena aneh yang saya sebut \textbf{"Negative Economic Spread"}.

Ternyata, agar angka simulasi cocok dengan slide BKF, kita harus menggunakan asumsi:
\begin{itemize}
    \item \textbf{Kenaikan Gaji ($s$): 7.87\% per tahun.} (Sangat agresif).
    \item \textbf{Hasil Investasi ($i$): 6.53\% per tahun.} (Konservatif).
\end{itemize}

Ini adalah kondisi \textit{anomaly} di mana beban hutang (gaji) tumbuh lebih cepat daripada kemampuan membayar (investasi).

\section{Hasil Validasi Akhir}

Setelah menerapkan model \textit{Joint Life} dan parameter \textit{Negative Spread} tersebut ke dalam kode Python, saya mendapatkan hasil sebagai berikut:

\begin{table}[htbp]
  \centering
  \caption{Hasil Validasi Akhir (Rigorous Math)}
  \begin{tabular}{lrrr}
    \toprule
    \textbf{Metrik} & \textbf{Hitungan Kita} & \textbf{Target Slide 17} & \textbf{Akurasi} \\
    \midrule
    Gaji Awal & Rp 2.500.000 & Rp 2.500.000 & Tepat \\
    \textbf{Total Aset} & \textbf{Rp 250.067.107} & \textbf{Rp 249.783.000} & \textbf{99.8\%} \\
    \textbf{Total Liabilitas} & \textbf{Rp 562.203.745} & \textbf{Rp 561.752.000} & \textbf{99.9\%} \\
    \textbf{Defisit (Gap)} & \textbf{(Rp 312.136.637)} & \textbf{(Rp 311.969.000)} & \textbf{Valid} \\
    \bottomrule
  \end{tabular}
\end{table}

Dengan tingkat akurasi di atas 99\%, dapat disimpulkan bahwa kita telah berhasil memecahkan logika perhitungan BKF. Defisit masif tersebut nyata secara matematis jika kita menggunakan asumsi bahwa gaji akan naik tinggi namun hasil investasi dana pensiun stagnan.

% \section{Teori Dasar}
% Tuliskan teori Anda di sini. Gunakan environment theorem untuk teorema.

% % Contoh penggunaan environment teorema (theorem)
% \begin{theorem}{Nama Teorema}{label_teorema}
%     Isi dari teorema matematika atau pernyataan penting.
% \end{theorem}

% \section{Analisis Data}
% Anda bisa memasukkan tabel, gambar, atau persamaan matematika di bagian ini.

% % Contoh persamaan matematika
% \begin{equation}
%     E = mc^2
% \end{equation}

% % --- Daftar Pustaka (Opsional) ---
% % Hilangkan komentar di bawah ini jika Anda sudah memiliki file reference.bib
% % \printbibliography[heading=bibintoc, title=Daftar Pustaka]

% % --- Lampiran (Appendix) ---
% \appendix
% \chapter{Lampiran A}
% Isi lampiran jika ada.

\end{document}