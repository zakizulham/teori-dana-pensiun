\documentclass[lang=en,10pt]{elegantbook}

% --- Informasi Metadata Buku ---
\title{Laporan Validasi Pensiun}
\subtitle{Log Percobaan Cek Balance}

\author{Muhammad Zaki Zulhamlizar}
\institute{Universitas Indonesia}
\date{\today}
\version{1.0}

% Info bio kustom (Kiri: Judul Label, Kanan: Isi)
\bioinfo{Info Khusus}{Percobaan Validasi Pensiun dengan \\ Cek Balance banyak eksperimen dan hasil yang menarik.}

% Kutipan atau info tambahan di bawah info penulis
\extrainfo{Mata Kuliah Teori Dana Pensiun.}

% Kedalaman daftar isi (3 artinya sampai sub-subsection)
\setcounter{tocdepth}{3}

% --- Gambar Sampul dan Logo ---
% Pastikan file gambar ini ada di folder project Anda
\logo{muscle-cat.png}
\cover{cover.jpg}

% --- Perintah Dokumen ---
\usepackage{array}
\newcommand{\ccr}[1]{\makecell{{\color{#1}\rule{1cm}{1cm}}}}
\graphicspath{{figure/}}

\renewcommand{\contentsname}{Daftar Isi}
\renewcommand{\chaptername}{Bab}
\renewcommand{\figurename}{Gambar}

% --- Kustomisasi Warna ---
% Ubah warna pita oranye pada halaman judul (opsional, hilangkan komentar untuk mengaktifkan)
\definecolor{customcolor}{RGB}{32,178,170}
\colorlet{coverlinecolor}{customcolor}

\begin{document}

% --- Bagian Depan (Frontmatter) ---
\maketitle      % Membuat halaman judul
\frontmatter    % Penomoran halaman romawi

\tableofcontents % Membuat Daftar Isi

% --- Bagian Utama (Mainmatter) ---
\mainmatter     % Penomoran halaman arab (1, 2, 3...)

% =====================================================
% MULAI MENULIS DI SINI
% =====================================================

\chapter{Brainstorming dulu}

Apa sih yang gw lakuin?\\
Jadi, gw ngelakuin eksperimen buat validasi dana pensiun pake metode cek balance. Jadi, gw ngecek balance dari dana pensiun yang ada buat liat apakah sesuai sama ekspektasi atau enggak.

\vspace{0.5cm}
Jadi, ngapain? \\
Spesifiknya gw akan melakukan validasi angka-angka pada Slide 17 (``Permasalahan Program Jaminan Pensiun'') dan Slide 16 (``Permasalahan Program Jaminan Hari Tua''), kita perlu membedah ``mesin'' aktuaria di baliknya.

\vspace{0.25cm}

Gw akan bandingkan dua sisi neraca:
\begin{itemize}
    \item \textbf{Sisi Aset (Accumulated Fund)}: Berapa uang yang terkumpul dari iuran (mirip prinsip Defined Contribution).
    \item \textbf{Sisi Aset (Accumulated Fund)}: Berapa uang yang seharusnya dibayar ke peserta berdasarkan perhitungan aktuaria (mirip prinsip Defined Benefit).
\end{itemize}

Kalau kedua sisi ini nggak balance, berarti ada yang salah di perhitungan atau asumsi yang dipakai.

\vspace{0.25cm}

Ketimpangan antara kedua sisi ini bisa ngasih insight tentang:
\begin{itemize}
    \item Apakah iuran yang dikumpulkan cukup buat bayar manfaat pensiun.
    \item Apakah asumsi aktuaria (seperti tingkat bunga, mortalitas, dsb) realistis.
    \item Potensi risiko pendanaan jangka panjang dari program pensiun tersebut.
    \item Ketimpangan (Unfunded Liability) yang mungkin ada.
\end{itemize}

\vspace{0.1cm}

Dengan ngecek balance ini, kita bisa validasi apakah program pensiun itu sehat secara finansial atau perlu penyesuaian.

\vspace{0.25cm}

% --- Bab 1 Decostruction & Theory ---
Sebelum masuk ke eksperimen, gw harus menetapkan asumsi matematis berdasarkan teori aktuaria standar yang kemungkinan besar digunakan oleh BKF (Badan Kebijakan Fiskal) dalam slide tersebut.

\section{Validasi Jaminan Pensiun}

Gw mulai dengan validasi jaminan pensiun. Berdasarkan slide 17, gw asumsikan beberapa parameter dasar:
\begin{itemize}
    \item Tingkat bunga diskonto: 7\% per tahun
    \item Mortalitas: Tabel mortalitas standar (misalnya TMI-2019)
    \item Iuran: 3\% dari gaji peserta
    \item Manfaat pensiun: Dihitung berdasarkan rumus tertentu (misalnya 1.5\% per tahun masa kerja)
    \item usia pensiun: 58 tahun
    \item usia masuk kerja: 30 tahun
    \item periode pengamatan: 30 tahun
    \item Frekuensi pembayaran: tahunan
    \item Distribusi gaji: Pertumbuhan gaji 5\% per tahun
    \item Formula Manfaat (Juli 2015 - Sekarang):
\end{itemize}
    
    \begin{equation}
        \text{Benefit} = 1\% \times \text{Masa Iur} \times \text{Rata-rata Upah tertimbang}
    \end{equation}

\begin{itemize}
    \item \textbf{Valuasi Liabilitas:} Nilai Manfaat di tabel slide adalah Actuarial Present Value (APV) dari anuitas seumur hidup pada usia pensiun.
\end{itemize}

% Persamaan hanya menarik 12 tahun iuran
\begin{equation}
    APV = \text{Manfaat Tahunan} \times \ddot{a}_{x}^{12}
\end{equation}

\begin{itemize}
    \item \textbf{Valuasi Aset:} Akumulasi iuran (3\% dari upah) yang dikembangkan dengan tingkat pengembalian investasi (investment return).
\end{itemize}

\begin{equation}
    FV = \sum_{t=1}^{n} \text{Iuran Tahunan} \times (1 + r)^{n-t}
\end{equation}

\vspace{0.5cm}

\section{Implementasi Eksperimen}

gw buat skrip python untuk ngitung aktuaria untuk reverse engineering nilai-nilai di slide. gw taro di `src/'.

\vspace{0.5cm}

Angka pada Slide 17 sangat bergantung pada asumsi makro ekonomi jangka panjang yang tidak ditulis eksplisit di slide.
\begin{itemize}
    \item Jika Salary Increase tinggi, Nilai Manfaat akan meledak (karena rata-rata upah naik).
    \item Jika Investment Return rendah, Akumulasi Dana akan mengecil.
    \item Ketimpangan yang masif di Slide 17 (-311 juta unfunded untuk gaji 2.5 juta) mengindikasikan bahwa asumsi \textbf{Kenaikan Gaji $>$ Return Investasi} atau \textbf{Discount Rate Liabilitas sangat rendah} (sangat konservatif).
\end{itemize}

\chapter{Menganalisis Slide 17}

\section{Investigasi Awal: Pendekatan Single Life}

Langkah pertama dalam eksperimen ini adalah mencoba mereproduksi angka pada Slide 17 menggunakan asumsi standar aktuaria sederhana: \textit{Single Life Annuity} (Hanya peserta, tanpa pasangan) dengan tingkat kenaikan gaji dan investasi yang moderat.

Hasil simulasi awal menggunakan Python memberikan data sebagai berikut:

\begin{table}[htbp]
  \centering
  \caption{Perbandingan Awal (Single Life Assumption)}
  \begin{tabular}{lrrr}
    \toprule
    \textbf{Item} & \textbf{Hitungan Kita (Rp)} & \textbf{Target BKF (Rp)} & \textbf{Status} \\
    \midrule
    Upah Awal & 2.500.000 & 2.500.000 & \textcolor{green}{Match} \\
    Akumulasi Aset & 151.960.069 & 249.783.000 & \textcolor{red}{Too Low} \\
    Liabilitas (PV) & 318.986.765 & 561.752.000 & \textcolor{red}{Too Low} \\
    Defisit & (167.026.696) & (311.969.000) & \textcolor{red}{Mismatch} \\
    \bottomrule
  \end{tabular}
\end{table}

\textbf{Analisis Kegagalan Awal:}
\begin{itemize}
    \item \textbf{Sisi Aset:} Nilai hitungan kita jauh lebih rendah. Ini mengindikasikan bahwa asumsi kenaikan gaji (\textit{salary scale}) yang digunakan BKF jauh lebih agresif daripada asumsi standar (5\%), sehingga iuran nominal yang masuk lebih besar.
    \item \textbf{Sisi Liabilitas:} Nilai target BKF hampir dua kali lipat dari hitungan kita. Ini adalah indikator kuat bahwa asumsi \textit{Single Life} tidak valid. Program Jaminan Pensiun (JP) di Indonesia mewajibkan manfaat turunan bagi janda/duda.
\end{itemize}

\section{Koreksi Logika Aktuaria: Reversionary Annuity}

Berdasarkan kegagalan pada percobaan awal, model matematika diperbarui menjadi model yang lebih \textit{rigorous} (ketat). Kita meninggalkan asumsi \textit{Single Life} dan beralih ke \textit{Joint Life Last Survivor} dengan fitur \textit{Reversionary Benefit}.

Rumus valuasi liabilitas diperbarui menjadi:

\begin{equation}
    PV_{Liabilitas} = B_{tahunan} \times \ddot{a}_{reversionary}
\end{equation}

Dimana faktor anuitas reversionary didefinisikan sebagai kombinasi antara anuitas peserta dan manfaat janda/duda:

\begin{equation}
    \ddot{a}_{reversionary} = \left(\ddot{a}_x - \frac{11}{24}\right) + Pct_{survivor} \times (\ddot{a}_y - \ddot{a}_{xy})
\end{equation}

Keterangan:
\begin{itemize}
    \item $\ddot{a}_x$: Anuitas seumur hidup peserta (dikoreksi Woolhouse $-11/24$ untuk pembayaran bulanan).
    \item $Pct_{survivor}$: Persentase manfaat janda (Asumsi regulasi: 50\%).
    \item $(\ddot{a}_y - \ddot{a}_{xy})$: Probabilitas istri hidup \textbf{setelah} suami meninggal.
\end{itemize}

Dengan asumsi demografi istri 5 tahun lebih muda dari peserta, faktor pengali liabilitas meningkat signifikan, mendekati pola angka BKF.

\section{Reverse Engineering Parameter Ekonomi}

Setelah struktur logika matematika diperbaiki, tantangan berikutnya adalah menemukan \textit{hidden parameters} (asumsi ekonomi) yang digunakan BKF. Menggunakan algoritma \textit{solver} numerik (\textit{Brent's method}), kita mencari kombinasi \textit{Salary Increase} ($s$) dan \textit{Investment Return} ($i$) yang menghasilkan angka presisi.

\subsection{Temuan: Negative Economic Spread}

Hasil \textit{reverse engineering} menemukan fakta mengejutkan yang menjadi akar masalah defisit struktural pada Slide 17. BKF menggunakan asumsi di mana gaji tumbuh lebih cepat daripada hasil investasi.

\begin{theorem}{Negative Spread}{neg_spread}
    Defisit masif terjadi karena tingkat kenaikan gaji ($s$) lebih tinggi daripada tingkat pengembalian investasi ($i$).
    \begin{equation}
        s > i \quad \rightarrow \quad 7.87\% > 6.53\%
    \end{equation}
\end{theorem}

Detail parameter yang ditemukan:
\begin{itemize}
    \item \textbf{Kenaikan Gaji ($s$):} 7.87\% per tahun. (Menyebabkan rata-rata upah tertimbang melonjak drastis).
    \item \textbf{Return Investasi ($i$):} 6.53\% per tahun. (Aset gagal mengejar pertumbuhan kewajiban).
    \item \textbf{Diskon Liabilitas ($v$):} 5.70\% per tahun.
\end{itemize}

\section{Hasil Reproduksi Akhir}

Dengan menggunakan formula \textit{Reversionary Annuity} dan parameter \textit{Negative Spread} di atas, kita berhasil mereproduksi angka Slide 17 dengan tingkat akurasi yang sangat tinggi (selisih $<0.5\%$).

\begin{table}[htbp]
  \centering
  \caption{Validasi Akhir (Rigorous Math + Forensik Ekonomi)}
  \begin{tabular}{lrrr}
    \toprule
    \textbf{Metrik} & \textbf{Hitungan Kita} & \textbf{Target Slide 17} & \textbf{Akurasi} \\
    \midrule
    Gaji Awal & Rp 2.500.000 & Rp 2.500.000 & Tepat \\
    Rata-rata Upah & Rp 10.218.117 & (Implisit) & - \\
    \textbf{Total Aset} & \textbf{Rp 250.067.107} & \textbf{Rp 249.783.000} & \textbf{99.8\%} \\
    \textbf{Total Liabilitas} & \textbf{Rp 562.203.745} & \textbf{Rp 561.752.000} & \textbf{99.9\%} \\
    \textbf{Defisit (Gap)} & \textbf{(Rp 312.136.637)} & \textbf{(Rp 311.969.000)} & \textbf{Valid} \\
    \bottomrule
  \end{tabular}
\end{table}

\textbf{Kesimpulan Bab Ini:}
Angka "mengerikan" pada Slide 17 bukan kesalahan hitung, melainkan representasi matematis dari kondisi ekonomi di mana janji manfaat (berbasis gaji) berlari lebih cepat daripada kemampuan aset (investasi) untuk berkembang. Inilah yang disebut sebagai \textit{Structural Unfunded}.

% \section{Teori Dasar}
% Tuliskan teori Anda di sini. Gunakan environment theorem untuk teorema.

% % Contoh penggunaan environment teorema (theorem)
% \begin{theorem}{Nama Teorema}{label_teorema}
%     Isi dari teorema matematika atau pernyataan penting.
% \end{theorem}

% \section{Analisis Data}
% Anda bisa memasukkan tabel, gambar, atau persamaan matematika di bagian ini.

% % Contoh persamaan matematika
% \begin{equation}
%     E = mc^2
% \end{equation}

% % --- Daftar Pustaka (Opsional) ---
% % Hilangkan komentar di bawah ini jika Anda sudah memiliki file reference.bib
% % \printbibliography[heading=bibintoc, title=Daftar Pustaka]

% % --- Lampiran (Appendix) ---
% \appendix
% \chapter{Lampiran A}
% Isi lampiran jika ada.

\end{document}